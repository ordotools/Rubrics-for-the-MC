% !TEX program = lualatex

% \documentclass[letterpaper]{article}
\documentclass[letterpaper]{report}

\title{Pontifical Mass at the Faldstool}

\usepackage{rubrics}


\usepackage{csquotes}
\usepackage[backend=biber,style=reading,style=verbose-ibid]{biblatex}
\addbibresource{~/Library/texmf/bibtex/bib/rubrics.bib}
%\bibliography{rubrics}

\begin{document}

\maketitle

\singleChap{Solemn Pontifical Mass at the Faldstool}{

\section{Vesting in the Sactisty}

\rubric The ceremony start about 25 minutes before the scheduled time. D and SD
vest while B reads the vesting prayers, book and candle bearers assisting, AP
assisting B. After the B is finished reading the book and candle bearers
depart. B removes cross and manteletta, and sits with biretta for his
handwashing by ACs, who wash B's hands standing, AP assisting, ministering ring and towel.

\rubric B stands after the handwashing and D and SD assist B with vesting. B
sits after donning his dalmatic. AP imposes ring after B dons chasuble. D
imposes the miter and D and SD get maniples.

\rubric Incense is imposed by B, D attending, TH standing. B recives crosier from SB, all stand
and MC1 signals for the procession to start.

\section{Confession, Introit, Gloria}

\rubric AP walks to the right of B, D walks to the left of B, SD walks before
B, holding the Evangeliarum.

\rubric All bow to the choir, D removes miter, SD hands book of Gospels to MC2,
and Mass begins.

\rubric SD imposes maniple at \textit{Indulgéntiam}, D backing up.

\rubric After kissing the altar, B turns to his left and kisses the book of
Gospels, held by SD, the text pointed out by AP. MC2 takes the book from SD
and goes to the Epistle side to prepare book- and candle-bearers.

\rubric B incenses the altar assisted by D and SD. SD\footnote{S.R.C. n. 2138}
imposes miter and D incenses B with three swings. AP stands by the faldstool.

\rubric B receives crozier, and goes \textit{per breviorum} to the faldstool. D
and SD ascend the predella after incensing B and flank him, D walking in front
of SD and B and turning towards B to stand on his right side, SD walking
straight to B's left side.

\rubric B sits for a few moments, then crozier away, D removes miter; B turns
to the left to face the altar and reads the Introit. Candle-bearer stands to
the right of the book bearer. D stands to the right of B, SD to his
left\footcite[See note 5, p. 241,][]{stehle}. AP stands to the right of B and
near the book.

\rubric B recites \textit{Kyrie} with the ministers, turns to the right to face
the people, sits and receives miter from D. D \& AP impose gremiale. D before
B, AP to D's left and SD to D's right, all three bow deeply to B and swing
towards the sedilia, so that AP sits closest to B. This is the usual manner in
which the ministers bow and then sit.

\rubric Minsters rise after the last \textit{Kyrie}\footcite[][227, p.
241]{stehle}, bow to B, gremiale and miter removed by D and AP; D and SD stand
\textit{unum post alium}. B intones the \textit{Gloria}, AP holding the
Canon\footnote{The assistant priest holds the book whenever the bishiop is
singing, otherwise the book bearer holds the book.}; D and SD flank B for the
recitation. B and ministers sit for the remainder of the sung \textit{Glória}.

\rubric At the end of the \textit{Glória} B and ministers rise, D and SD stand
\textit{unum post alium} and B sings \textit{Dóminum vobíscum.} B sings the
orations, AP holding the Missal. SD receives the Book of Epistles from MC2
after the last \textit{Per Dóminum}, and stands some distance from B, where the
Epistle is usually read.

\section{Epistle and Gospel}

\rubric B sits with miter and gremiale; D \& AP sit. SD bows to B and sings the
Epistle; afterwards he repeats the reverence, goes the middle and genuflects,
and kneels to receive the B's blessing.

\rubric While SD is making his reverences after the Epistle, MC2 sends the book
and candle bearers to the faldstool. After SD receives the blessing from B, he
hands the book of Epistles to MC2, and receives the Missal from book bearer and
kneels before B (candle bearer to his right) and B reads the Epistle, Gradual,
Tract, etc. B recites the \textit{Munda cor} and reads the Gospel.

The assistant priest and the deacon meanwhile stand, turn the pages of the
Missal, etc. When the bishop has finished the Gospel, the subdeacon gives up the
Missal and stands at a convenient distance, facing the bishop.

\rubric Toward the end of the singing of the Gradual (Sequence), the deacon
receives the Book of Gospels and carries it to the altar with the prescribed
reverences. He genuflects on the platform, descends direct to the right of the
bishop, takes the boat from the thurifer (who has meanwhile come to the bishop
with the acolytes), and ministers incense. 

\rubric Kneeling on the edge of the platform, he says \textit{Munda,}etc. takes
the book, genuflects, and goes direct to the faldstool. Kneeling before the
bishop, he says \textit{Jube domne benedicere}, receives the blessing and
kisses the ring.\footcite[Stehle writes ``The subdeacon, the thurifer and the
acolytes kneel while the deacon the blessing.'' This requires more
research.][]{stehle}

\rubric D and the rest of teh procession go with the prescribed reverences to
the place where the Gospel is usually sung. The master of ceremonies removes
the gremial and the miter. The bishop rises and turns toward the deacon. The
assistant priest stands at the left of the bishop. After the Gospel, the
subdeacon without making any reverences carries the book to the bishop,
who kisses the text; the deacon incenses the bishop with three swings.

\section{From the Credo to the Offertory}

\rubric The bishop turns toward the altar and intones the \textit{Credo}, the
deacon and the subdeacon observing what was prescribed at the \textit{Gloria}
(227). B sits with the gold miter.

\rubric At \textit{Crucifxus,} the deacon rises, carries the burse to the altar
and spreads the corporal, observing what was prescribed (91). 

\rubric At the end of the \textit{Credo}, the ministers rise; the deacon removes the
gremial and the mitre. The bishop rises and, facing the people, sings Dominus
vobiscum, then turns (\textit{per latus sinistrum}) to the altar and sings Oremus from
the Missal, held by the book-bearer, While he reads the Offertory, the
assistant priest and the deacon stand at his right, the subdeacon at his left.

\rubric B turns toward the people (\textit{per latus dexterum}), sits, and
receives the precious mitre and the gremial from the deacon. AP then removes
the ring, \textit{cum osculis}, and D removes the glove of the right hand, the
subdeacon removes the glove of the left hand \textit{cum osculis} (272). The
bishop washes his hands. The assistant priest ministers the towel, and replaces
the ring: then takes the Missal, the Canon and the missal-stand to the altar
with the aid of the book-bearer (92).

\section{Offertory, Incensation, Preface}

\rubric While the bishop is washing his hands, the subdeacon, accompanied by
the second master of ceremonies, goes from the bench directly to the credence,
where the humeral veil is put on his shoulders by the acolytes. With his left
hand he takes the chalice, extends the right end of the veil over the pall,
places his right hand lightly over the veil and arranges to arrive at the altar
at the same altar time at the as the bishop. He places the chalice on the
epistle side and removes the end of the veil carefully.






}

%\bibliographystyle{plain}
\printbibliography
    
\end{document}
