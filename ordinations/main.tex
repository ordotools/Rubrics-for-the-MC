\documentclass[letterpaper]{report}
\title{Ordinations}
% \date{March 24-30, 2024}

\usepackage{rubrics}
% \usepackage{booktabs}
\usepackage{csquotes}
\usepackage[backend=biber,style=reading,style=verbose-ibid]{biblatex}
\addbibresource{~/Library/texmf/bibtex/bib/rubrics.bib}
% \usepackage{enumitem}
% \usepackage{import}

\newcommand{\pbr}[0]{\textit{per breviorum}}

\begin{document}

\maketitle

% \chap{Preparation}{}



\chap{Ceremony}{

    \section{Procession}

    \section{Tonsure}

    \section{Minor Order}

    Most in this section is taken from Fr. Cekada's notes.

    \rubric After the \textit{Gloria}, MC2 moves the faldstool to the middle of
    the predella. B sits and receives the gold miter and the crosier. B rises
    and proceeds to the foot of the altar flanked by D \& SD.

    \rubric B ascends with ministers, and sits in the faldstool with gremiale,
    D \& SD change places behind B. MC2 arranges the ordinands in the sanctuary
    before B, some distance from the foot of the altar. Ordinands carry
    \textit{extinguished} candles.

    \rubric Archdeacon says \textit{Accedat qui ordinandus est ad\dots.} and
    then reads the names of the ordinands, pausing slightly after each one.
    During this pause the ordinand names says \textit{Adsum} and takes one step
    forward. Once all the ordinands have been called, MC2 signals them to kneel
    in place.

    \rubric BB \& CB approach B with the Pontifical. B recites the instruction
    prescribed for the order. After the instruction, MC2 takes the candles from
    the ordinands.

    \rubric During the instruction, MC1 prepares the matter for each of the
    orders: Keys on a tray for Poter; Missal or Epistolarium for Lector; Ritual
    for Exorcist; Candlestick with candle and an empty wine cruet for Acolyte.
    After the instruction, MC1 brings the proper matter to D, who hands it to
    B.

    \rubric Once B has the matter, MC2 signals the two most senior ordinands to
    rise and approach B. Before ascending, the ordinands genuflect to the cross
    on the floor and bow profoundly to B, then ascend and kneel on the top
    step, close enough to B to be able to easily reach the matter. The
    ordinands touch the matter with their right hands and then B recites the
    form of conferral of order. The ordinands then descend, when reaching the
    floor first bowing profoundly to B and then genuflecting to the
    cross.\footnote{While it is permissible, for the sake of preserving time,
    to have the next ordinands make these reverences with the ones descending,
it seems to be better to reserve such a shortcut to an ordinary.} The ordinands
kneel in their places, and the next ascend, until all have been ordained.

    \rubric \textbf{For Porter}, the ordinands are led by MC2 to the church
    door, where they open and close the church door and then ring the church
    bell.

    \rubric B stands with miter and reads the blessing over the ordinands.

    \rubric B sits, D removes miter, B stands to face the altar. D \& SD need
    not change places. B chants \textit{Oremus} D \& SD recite
    \textit{Flectamus\dots, Levate} in the usual way, while everyone
    genuflects. All three turn to face the ordinands and B recites the prayer(s).

    \rubric While B's miter is being removes, BB \& CB go behind B and stand
    with the Pontifical open.\footnote{It is proper that the bishop have the
        book before him whenever the text he is reciting is contained therein,
        but an exception can be made in this case when space does not allow, or
    some other reason that would make the movement awkward, rushed or otherwise
lack grace.} After \textit{Levate}, they return to stand before and to the left
of B.

    \rubric After the prayer(s), MC2 signals the ordinands to rise. All
    ordinands first bow to B then genuflect to the cross and go to thier
    places.

    \rubric BB \& CB away; D removes miter and B and ministers descend to the
    foot of the altar, reverence and go to the faldstool to resume Mass from
    the \textit{Pax vobis} before the Collects. Meanwhile, MC2 takes the
    faldstool from the predella and places it in its usual position.


    \section{Subdiaconate}

    \rubric B sits after the Collects and receives the gold miter from D and
    crosier. B rises and prodeeds to the foot of the altar flanked by D \& SD. 


    \section{Diaconate}

    \rubric The ordination of deacons occurs immediately after B reads the
    Epistle.

    \rubric The bishop sits in the faldstool with miter and gremiale. The
    archdeacon calls all those who are to be ordained to the altar, and they
    come two by two as they are called. Then the archdeacon calls each by name.
    Each replies adsum as his name is called (dalmatic and stole over his left
    arm\footcite[The ordinands can also hold the stoles in their left hands,
    and have their dalmatics draped over their left arms.][]{levav:ordinations}
    and extinguished candle\footcite[It appears that the candles could be lit,
    or are customarily lit in some places.][]{levav:ordinations} in his right
    hand), steps forward, genuflects to the cross and bows deeply to the bishop
    (to an ordinary, this is a single genuflection made to both) and goes to
    his position. 

    \rubric MC2 positions all the ordinands before the bishop in a semicircle,
    or in lines if there are too many of them.

    \rubric All of the ordinands kneel. Book and candle bearer approach the
    bishop, and the archdeacon reads \textit{Excellentissime} etc. The bishop
    interrogates the archdeacon, who answers, to which the bishop replies
    \textit{Deo gratias}. 

    \rubric The bishop remains seated and reads \textit{Auxiliante Domino} to
    the clergy and people, after which the archdeacon returns to his place in
    choir.\footcite[Le Vavasseur has the archdeacon return to his place
    immediately after \textit{Deo gratias.}][]{levav:ordinations}

    \rubric The bishop, after a little pause, reads addresses the exhortation
    \textit{Provehéndi} to the ordinands. After he has finished, all the
    ordinands respond Amen and MC2 takes their candles. 

    \rubric If there be no ordination to the subdiaconate, the archdeacon calls
    all the ordinands to the priesthood to come before the bishop two by two,
    chasubles folded over their arms and without candles, genuflect to the
    cross and bow to the bishop. MC2 signals all the ordinands to prostrate,
    the deacons by the Gospel corner and the priests by the Epistle corner, and
    the archdeacon leaves. After the litany, all rise and the archdeacon,
    facing away from the altar, says Rededant qui ordinandi sunt presbyteri,
    and the ordinands to the priesthood return to their places, two by two,
    making the proper reverences, and the ordinands to the diaconate kneel
    before the bishop. The archdeacon returns to his place.

    \rubric The bishop, siting with miter and gremiale, in a clear loud voice
    says \textit{Commúne votum} and blesses the ordinands with his right hand.
    At the end his assistants reply \textit{Amen}. 

    % NOTE: check the recto tono
    \rubric Gremiale removed, the bishop rises with the miter, faces the
    ordinands and sings recto tono \textit{Orémus, fratres carissimi} with
    hands joined until \textit{regnat Deus}, and blesses the ordinands with his
    right hand. The bishop then sits, his miter is removed, he rises and raises
    his voice in the tone of a ferial preface, his hands extended before his
    breast and sings the ordination preface until the words \textit{quæ sunt
    agénda concédere}. 

    % NOTE: There are quite a few differences here between the authors
    \rubric MC2 has the ordinands rise and stand before the bishop \textit{unus
    post alium}. The first to be ordained genuflects on the floor, ascends and
    kneels on the edge of the footpace. The bishop, still standing and with his
    left hand on his breast, imposes his right hand on the head of the
    ordinand, still wearing his gloves and in a low but clear voice, without
    singing says \textit{Accipe Spiritum Sanctum,} etc. The ordinand rises,
    turns to his left and descends to the floor where he bows to the bishop,
    genuflects and goes to his place and stands. The next in line approaches,
    makes his reverences, goes up, etc. until the form has been repeated over
    all singly.

    \rubric The deacons all kneel in their places. The bishop continues the
    preface to the end, his right hand extended towards the ordained, his left
    hand remaining on his breast. This extension is of great importance. when
    he reads \textit{per eundem, etc.}, he lowers his voice (submissa voce) and
    joins his hands before his breast. 

    \rubric B sits in the faldstool and receives his miter from D. All of the
    deacons rise and stand \textit{unus post alium}. Before ascending, they
    pass their stoles and dalmatics to the MCs, who give the stoles to the
    deacon of the Mass and keep the dalmatics. 

    % NOTE: does this sign of the cross come after the Amen? #103 Le Vav.
    \rubric The new deacons go up one by one with the same reverences as
    before. B receives the stole from D, and puts it over the left shoulder of
    the new deacon while saying \textit{Accipe stolam} etc., blesses the deacon
    with his right hand, who responds \textit{Amen}. The MC at the right of the
    deacon fixes the stole under the new deacon's right arm.

    \rubric The MC with the dalmatic hands it to B, and D \& SD assist B in
    vesting the new deacon. B meanwhile says \textit{Induat te Dominus} etc.,
    and the new deacon replies \textit{Amen}. The first deacon rises, turns to his
    left and descends to the floor, reverences as before and goes to his spot,
    and the rest of the new deacons receive their vestments in the same way.

    \rubric If there is only one tunic, B puts it on each new deacon up to his
    shoulders, then removes it, and puts it fully on the last of the new
    deacons.

    \rubric After everyone is vested, MC1 gives the book of Gospels to D. D
    gives it to B with the usual kisses,\footcite[p. 116, n.
    104.1][]{levav:ordinations} and B holds it in his lap. B, who holds it in
    his hands. The new deacons line up in two lines. Two by two, the new
    deacons genuflect on the floor, bow to B, ascend, kneel before B and touch
    the book with their right hands. B reads \textit{Accípite potestátem} etc.
    (in the plural) to which they respond \textit{Amen}\footnote{S.R.C. 2682,
    ad 6.}. The deacons rise and turn towards one another, descend to the
    floor, reverence B and the cross and return to their places. The rest of
    the deacons touch the book in the same manner. If the number of deacons is
    odd, three go together in the final group. D takes the book from B with the
    usual kisses, and gives it to AC1, who returns it to the credence table.

    \rubric All of the new deacons kneel in their places. D removes miter and B
    turns to the altar and sings \textit{Oremus}; the ministers, also turned
    towards the altar, sing \textit{Flectamus genua; Levate} and genuflect. B
    and ministers turn towards the new deacons and B (hands joined) sings the
    oration \textit{Exaudi, Domine}, blessing the new deacons with his right
    hand and says \textit{Orémus. Dómine sancte}. BB \& CB leave and B sits
    with miter.

    \rubric During the oration, Archdeacon stands on the Gospel side of the
    altar at MC2's signal and says clearly \textit{Ad loca
    vestra}.\footnote{Alternatively, the archdeacon could say \textit{Recédant
    qui ordináti sunt Diáconi}. S.R.C. 2682 ad 4.} All the new deacons rise,
    genuflect to the altar, bow profoundly to B and return to their places.
    Archdeacon returns to his place. Book \& candles bearers retire, candle
    bearer placing the bugia on the altar. 

    % TODO: add the part about the singing for the choir

    \section{Continuation of Mass}

    \rubric B receives his crosier and returns to the faldstool.

    \section{Ordination of Priests}

    \rubric \dots

    \section{Continuation of Mass}

    \rubric The Gospel is sung by one of the new deacons, the book being held
    by the SD who sang the Epistle. After the Gospel, B kisses the book and is
    incensed by the AP.

    \rubric After reading the Offertory prayer, B sits and receives the
    precious miter. MC2 gives the ordained their candles and AC2 stands by B
    with a tray to receive the candles. B receives the linen gremiale from the
    ministers. The ordained come before B one by one, present the candle and
    then kiss B's hand.

    \rubric After the first oration after the \textit{Agnus Dei}, B kisses the
    altar with AP and gives AP the pax. AP distributes the pax to the choir,
    but not to the newly ordained. When B begins to give the pax to his
    assistants, all the new priests rise and MC2 leads the first (1P) of the
    new priests, first (1D) of the new deacons and first (1SD) subdeacon to the
    altar after the ministers of the Mass.

    \rubric 1P genuflects to the Blessed Sacrament, ascends the altar, kisses
    the altar with his hands touching the mensa, because he is concelebrating.
    He receives the pax from the bishop, bowing his head profoundly before and
    afterwards. Then he genuflects to the Blessed Sacrament, descends, and give
    the pax to the second new priest, who gives it to the third, etc.

    \rubric 1D and 1SD recieve the pax from B in the same manner as 1P, but
    kiss the altar with joined hands. 1D gives the pax to the second new
    deacon, etc. 1SD gives the pax to the second new subdeacon, etc., and give
    the pax to the first of the new acolytes, who gives the pax to the rest of
    the ordained, until all have received it.

    \section{Communion of the Ordained}

    \rubric After the consumption of the Precious Blood, MC2 lines up all the
    new priests two by two or four by four at the foot of the altar. B
    administers Communion to all the new priests, saying nothing but making the
    sign of the cross with the host, the new priests kissing his ring before
    receiving.

    \rubric B replaces the Blessed Sacrament on the corporal. MC2 gives the new
    deacon who chanted the Gospel a card with the chant of the
    \textit{Confiteor}. The new deacon rises and goes to the epistle side,
    where he stands on the bottom step, facing the gospel side. He makes a
    medium bow and sings the \textit{Confiteor}.

    \rubric B recites \textit{Misereatur} and \textit{Indulgentiam} and
    absolves with his right hand, as usual. After the \textit{Indulgentiam},
    the new deacon genuflects on the floor and returns to his spot.

    \rubric All the ordained rise and go to the foot of the altar in twos or
    fours for Communion. B recites the \textit{Ecce} as usual.

    \rubric For the new deacons and subdeacons, B recites: \textit{Corpus
    Domini nostri Jesu Christi, custodiat te in vitam æternam}, to which each
    replies \textit{Amen}, then kisses B's ring and receives the Blessed
    Sacrament.

    \rubric If none of the ordained be deacons, the \textit{Confiteor} is sung
    by the deacon of the Mass.

    \rubric If those who received minor orders and tonsure receive Communion, B
    says \textit{\dots custodiat te} and all respond
    \textit{Amen}.\footnote{S.R.C. pg. 131}

    \rubric There should be as many hosts consecrated as there are clerics who
    will communicate, and nobody else should receive Communion, that way there
    will be no hosts left over.\footcite[This seems to be a recommendation, not
    a command.][]{nabuco:pontificalis:1}

    \section{Penance}

    \rubric After the blessing, MC2 places the faldstool in the middle of the
    footpace. B retains his crosier and sits. Book and Candle Bearers approach
    and kneel to B's left. All of the newly ordained kneel in their places.

    \rubric B reads \textit{Filii dilectíssimi} and then joins the penance.
    After receiving the penance, the newly ordained reply
    \textit{Libénter.}\footnote{This is not rubrical, but is the Roman custom.}

}

\printbibliography

\end{document}
