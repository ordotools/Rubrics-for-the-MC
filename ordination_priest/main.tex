\documentclass{report}

\usepackage{rubrics}

\begin{document}

\chap{De Ordinatione Presbyteri}{

	\section{Procession to the altar}

	\rubric After the second to last verse of the sequence \textit{Lauda Sion
	Salvatorem} has been sung\footnote{The rule is that the ordination of
	priests takes place before the last verse of the last text that is read
	after the Epistle.}, the Bishop, once the gremiale has been removed, goes
	from the faldstool to the altar and sits on the faldstool, where he gives
	away the crozier and receives the gremiale.

	\section{The summons}

	\rubric Ordinand stands in front of B about three paces from the predella.

	\rubric AP standing on the predella, to the right of B, calls the ordinand
	saying \textit{Accédat qui ordinándus est ad órdinem presbyterátus.} He
	then calls the ordinand by name.

	\rubric Ordinand, with an extinguished candle in his right hand and his chasuble
	over his left arm, responds Adsum and takes two steps towards the altar. He
	genuflects, bows to the bishop and remains standing.

	\section{Postulatio}

	\rubric The book and candle bearers come before the bishop and stand at his
	right as the the Assistant Priest says \textit{Excellentissime ac
	reverendissme Pater, postulat sancta mater etc.}

	\rubric The Bishop responds: \textit{Scis illum dignum esse?}

	\rubric The Assistant Priest responds: \textit{Quantum humana fragilitas
	etc.} The Bishop says: \textit{Deo gratias.}

	\section{Consultation and election}

	\rubric The Bishop announces the election of the ordinand to the clergy and
	people, saying in a higher voice, not singing, \textit{Quoniam, fratres
	carissimi, etc.}, and at the end makes a short pause.

	\section{Exhortation}

	\rubric The Bishop, in a subdued yet audible voice, directing his speech to
	the ordinand, admonishes him saying \textit{Consecrandus, fili
	dilectissime, etc.} At the end the ordinand responds \textit{Amen}, and Mc2
	takes his candle and sets it aside. The book and candle bearers return to
	their places.

	\section{The Litany of the Saints}

	After the exhortation, the ordinand prostrates in the middle of the
	sanctuary. The Bishop, once the gremiale has been removed, kneels before
	the faldstool on which Mc1 places the Pontificale, and the choir begins the
	singing of the Litany. After the singing of the petition ut omnibus
	fidelibus defunctis, the bishop rises, and, having received the crozier,
	sings the blessings over the ordinand, to which the choir sings in
	response.

	\section{Imposition of hands}

	After the singing of the Litany, Mc2 invites the priests who are to impose
	hands to come to the altar, where he sees to it that they form a line
	beginning on the floor at the Gospel side of the altar, so that the
	ordinand is in the middle. They are vested in at least a stole of the color
	of the Mass, over a surplice.

	The priests who are present, beginning with the more senior, will impose
	hands after the Assistant Priest. The sacred ministers do not impose hands,
	even if they are priests, because they are functioning as ministers.
	Priests serving as MC do not impose, since they are otherwise occupied. The
	Bishop rises with miter. The ordinand rises at the signal given by Mc1,
	and, having made the reverences to the altar and the bishop, ascends to the
	altar, where he kneels before the Bishop on the edge of the predella. The
	Bishop imposes his extended hands on the head of the ordinand, physically
	touching him, saying nothing. The ordinand then descends and returns to his
	place, where he kneels.

	The Bishop, having imposed hands, remains standing, with his left hand on
	his chest and his right hand extended towards the ordinand. Mc1 sees to it
	that the priests who are present, one after the other, come before the
	ordinand and impose hands on the head of the ordinand physically touching
	him.

	Each of the priests, having imposed hands, returns to the place where he
	was before, with his right hand extended towards the ordinand. After the
	imposition of hands, the book and candle bearers approach the Bishop, who,
	with his right hand still extended towards the ordinand, sings the
	invocation Oremus, fratres carissimi, etc. Once he is finished singing, and
	not before, he joins his hands, and the other priests present follow his
	example. The Bishop sits, while the priests return to their places in
	choir. Once the miter has been removed, he again rises. He turns to the
	altar with the ministers, sings Oremus, after which the ministers sing
	Flectamus genua and Levate. The Bishop then, having turned back to the
	ordinand with joined hands, sings the oration Exaudi nos, etc., blessing
	the ordinand with his right hand.

	When he comes to the words Per omnia saecula saeculorum, he raises his
	voice to sing the ferial tone of the preface, and, with his hands extended
	before him, sings the preface. When he comes to the essential form, he
	pronounces it in a subdued yet audible voice. When he says Per eundem he
	joines his hands, and, in a clear yet subdued voice, reads the termination
	of the preface. The book and

	candle bearers move farther to the left of the Bishop.

	\section{Vestition}

	After the singing of the preface, the Bishop sits and receives the miter.
	Mc1 and Mc2 position themselves on the floor facing each other before the
	bishop, leaving room for the ordinand to ascend to the altar. The ordinand
	rises, gives his chasuble to Mc1, and, having made the required reverences,
	ascends to the altar, where he kneels on the edge of the predella. The
	Bishop, assisted by Mc1, arranges the ordinand’s stole in the form of a
	cross, saying Accipe jugum Domini etc., to which the ordinand replies
	nothing.

	Then, assisted by Mc1 and Mc2, the Bishop imposes the chasuble on the
	ordinand, taking care that the back part remain folded, saying Accipe
	vestem sacerdotalem etc. The ordinand responds Deo gratias, rises, turns by
	his left, descends to the floor, and having made the required reverences,
	returns to his place.

	\section{Blessing}

	After the vestition, the new priest kneels in his place and the bishop,
	once the miter has been removed, rises, and sings the oration Deus
	sanctificationum etc., blessing the new priest with his right hand.

	\section{The Anointing of the Hands}

	The book and candle bearers depart and Mc1 places a cushion before the
	faldstool and the Pontifical upon the faldstool. The Bishop, turning to the
	altar with his ministers, kneels, and everyone else in the church follows
	his example. The deacon removes his zuchetto. He intones the hymn Veni,
	Creator. After the singing of the first verse, the Deacon replaces the
	zuchetto, all rise, Mc1 removes the cushion, and the bishop sits on the
	faldstool, where he receives the miter from the deacon. The gremiale bearer
	genuflects on the floor and kneels on the edge of the predella, holding the
	linen gremiale. The Deacon removes the Bishop’s ring and his right-hand
	glove, while the Subdeacon removes the Bishop’s left-hand glove. The Deacon
	replaces the ring on the Bishop’s finger, and the gremiale bearer departs
	with the gloves. The Deacon places the linen gremiale on the Bishop’s lap,
	and, together with the Subdeacon, ties it to the faldstool. Mc1 gives the
	holy oil to the Deacon. The book and candle bearers position themselves to
	the left of the Bishop and kneel.

	The new priest rises, and, having made the required reverences, ascends to
	the predella, where he kneels. Mc1 takes the linen band from the right-hand
	side of the new priest’s cincture. The new priest presents his hands,
	opened yet joined at the sides, to the Bishop. The Bishop dips his right
	thumb into the vessel of holy oil held by the Deacon, and, with his left
	hand held under the hands of the new priest, anoints the new priest’s hands
	in the form of a cross, that is, he makes a line with the oil from the
	right thumb to the index finger of the left hand, saying Consecrare; then
	from the left thumb to the index finger of the right hand, saying et
	sanctificare. Then he anoints all of the remaining surface area of the
	palms of the new priest’s hands, saying digneris, Domine, manus istas per
	istam unctionem. Then saying et nostram benedictionem he blesses with the
	sign of the cross the hands of the new priest, to which the new priest
	responds Amen. The Bishop says Ut quaecumque benedixerint etc., to which
	the new priest responds Amen.

	The Bishop then closes the hands of the new priest, that is, he joins the
	palms together, and Mc1 at the right of the new priest binds his hands with
	the linen band.

	Once the new priest’s hands have been anointed, the Bishop purifies his
	fingers, and Mc1, having received the holy oil from the Deacon, returns it
	to the credence table.

	While the Bishop anoints the hands of the new priest, the choir sings the
	hymn Veni, Creator. If the length of the ceremony requires it, they may
	repeat the verses, except for the first and last.

	Clergy in choir stand during the anointing of the hands. Once the anointing
	of the hands has been finished, the choir sings the doxology of the Veni
	Creator, and all bow profoundly to the cross of the main altar, while the
	Bishop bows his head.

	\section{The traditio of the chalice and paten}

	Mc1 gives the Bishop a chalice containing wine mixed with water, together
	with a paten and host. The new priest touches the paten with his index
	fingers, and the cup of the chalice with his other fingers, so that he
	touches the cup of the chalice and the paten simultaneously, while the
	bishop says Accipe potestatem, etc.

	\section{Washing of the New Priest’s Hands}

	The new priest responds Amen, turns by his left, descends to the floor,
	where having made the required reverences, he goes to the place assigned,
	where he purifies and washes his hands. He returns to the altar, makes the
	required reverences, and returns to his places in choir.

	Washing of the Bishop’s Hands

	The acolytes approach the Bishop, and the Assistant Priest ascends to the
	predella to stand at the bishop’s right. The Bishop, his ring having been
	removed, purifies and washes his hands, the Assistant Priest administering
	the towel. The acolytes then depart, and the sacred ministers remove the
	linen gremiale, which the deacon gives to Mc1. The gremiale bearer, with
	the Bishop’s gloves on a tray, and having made the required reverences,
	kneels on the edge of the predella. The Deacon replaces the right-hand
	glove, and the Subdeacon the left-hand glove. The Assistant Priest then
	replaces the ring on the Bishop’s right hand.

	\section{Procession to Faldstool of the Mass and the Continuation of the
	Mass}

	The Bishop, having received his crozier, returns to the faldstool of the
	Mass, where, having given away the crozier and received the gremiale, reads
	the last verse of the sequence, which the choir sings. He then says the
	Munda cor meum and reads the Gospel as prescribed by the Ceremoniale.

	\section{The Offertory}

	The Bishop, having sung Dominus vobiscum and Oremus, reads the Offertory
	verse, for which the book and candle bearers stand before him. Then,
	sitting he receives the precious miter and the book and candle bearers
	depart. Meanwhile, the choir sings the Offertory verse.

	\section{Presentation of the Candle}

	The Deacon positions himself to the bishop’s right, the Subdeacon to his
	left. A server stands to the left of the Deacon. The ministers place the
	linen gremiale on the Bishop’s lap. Mc2 gives the new priest his lighted
	candle, who comes to the Bishop at the faldstool, reverencing the altar on
	the way. He reverences the Bishop, then kneels before him. He gives the
	Bishop his candle, kissing only the Bishop’s hand, his left hand on his
	chest. The Bishop, having received the candle, gives it to the Deacon, who
	gives it to the server, who in turn takes it to the credence table. The new
	priest rises, reverences the

	Bishop, and returns to his place, reverencing the altar on the way.

	\section{Washing of the Bishop’s Hands}

	After the presentation of the candle, the gremiale bearer comes to the
	faldstool, where the Bishop takes off his gloves and ring. The acolytes
	then approach and the bishop washes his hands. After the washing, the
	Bishop receives back his ring, and the linen gremiale is removed.

	\section{Procession to the Altar}

	The Bishop, once the gremiale has been removed and he has received the
	crozier, rises from the faldstool of the Mass flanked by the deacon and
	Assistant Priest, insignia bearers following behind.

	At the foot of the altar, the Bishop, having given away the crozier and
	after the miter has been removed, bows profoundly to the cross. The Deacon
	and Assistant Priest genuflect, and the Bishop ascends to the altar with
	the Deacon and Assistant Priest.

	At the beginning of the Offertory, the sacristans place two kneelers, each
	with a Missal, in the middle of the sanctuary, at a short distance from the
	altar, one of them on a direct line with the center of the altar, and the
	other to the right of the first.

	After the procession of the Bishop to the altar, Mc2 leads the new priest
	and his assistant priest to the kneelers.

	\section{Rules for the Bishop}

	The Bishop celebrates the Mass in the usual way, taking care that he say
	the parts which are usually said secretly somewhat higher, so that the new
	priest will be able to say all of the prayers with him, especially the
	words of Consecration. Rules for the new priest

	The new priest, kneeling, begins the concelebration of the Mass with the
	Bishop with the prayer Suscipe, sancta Pater and so continues until the
	Last Gospel exclusive, for which he stands, saying all the prayers in a
	clear yet subdued voice. He signs himself when the Bishop signs himself. He
	strikes his breast with him. However, he refrains from all inclinations
	which the Bishop makes, because he is already kneeling, and from the
	blessings over the oblata, incense, and water, keeping his hands always
	folded before his chest. He also refrains from the recitation of the prayer
	which accompanies the blessing of the incense. He recites in a subdued
	voice all the parts which the Bishop sings.

	\section{Incensation}

	The Deacon, having incensed the Bishop, incenses the Assistant Priest, then
	the priests in choir, then the Subdeacon. After that, the new priest rises,
	and the Deacon incenses him with a double swing.

	The thurifer incenses the Deacon, then other clergy in choir, the inferior
	ministers, and finally the people. The new priest’s assistant priest is not
	incensed, since he is otherwise occupied.

	\section{Torchbearers}

	At the Sanctus, the torchbearers come in with lit torches, and kneel in a
	line parallel to the altar behind the new priest.

	\section{Sanctus and Agnus Dei}

	The MCs must take care that the recitation of the Sanctus and Agnus Dei not
	be impeded by the singing of the choir, who sing after the recitation.

	\section{Pax}

	The Bishop gives the Pax to the Assistant Priest in the usual manner, who,
	after giving it to the Deacon, then gives it to the priests in choir but
	not to the new priest.

	After the recitation of the Agnus Dei, the new priest rises and Mc1 leads
	him to the altar. The new priest, having genuflected to the Blessed
	Sacrament on the lowest step of the altar, ascends to the altar, and kisses
	it, his hands on it because he is concelebrating. He receives the Pax from
	the Bishop, to whom he makes a profound bow of the head before and after.
	He then genuflects again to the Blessed Sacrament, descends to the floor,
	and returns to his place.

	After having given the Pax to the new priest, the Bishop waits for him to
	return to his place before reciting the second and third Communion prayers.

	\section{Communion of the New Priest}

	After the Bishop has consumed the Precious Blood, the Deacon covers the
	chalice. The new priest rises, genuflects, comes to the edge of the
	predella, and kneels.

	The Deacon uncovers the ciborium, and the Bishop genuflects with his
	ministers. The ministers switch sides, and the Deacon takes up the paten,
	and the Bishop gives the new priest the Blessed Sacrament, saying nothing,
	but only making the sign of the cross with the host. The new priest kisses
	the Bishop’s ring. The new priest rises, turns, descends to the floor,
	genuflects, and returns to his place. The ministers again switch sides, and
	the Deacon, facing the Gospel side, sings the Confiteor.

	After the singing of the Confiteor, the Bishop recites the Miseatur and
	Indulgentiam, absolving with his right hand. After the Misereatur the
	ministers cease bowing and switch sides.

	The sacred ministers genuflect together with the Bishop, the Bishop again
	takes up the ciborium, the Deacon the paten, and together with the Bishop
	turn to face the people. The Bishop, at the middle of the altar, having
	taken up one of the hosts, says Ecce agnus Dei etc. in the usual manner,
	and then distributes Holy Communion making the sign of the cross with the
	host and saying Corpus Domini nostril Jesu Christi, etc.

	When everyone who wishes to do so has received Holy Communion, the Bishop
	returns to the altar with his ministers.

	\section{Ablutions and Washing of the}

	Bishop’s Hands

	The Bishop, with the Deacon administering, receives the ablutions in the
	usual manner, purifying also the ciborium.

	The Assistant Priest carries the Missal stand with the Canon to the Epistle
	side of the altar, where he replaces it with the Pontifical and places the
	Canon in front of the tabernacle. The Subdeacon goes to the Gospel side of
	the altar, where he purifies the chalice and ciborium, and from whence he
	carries the covered chalice and corporal to the credence table.

	The Bishop, having received the miter, washes his hands in the usual
	manner, with the Assistant Priest administering the towel.

	The miter having been removed, the Bishop turns to the Missal stand, the
	ministers form a semicircle as at the Introit of a Solemn Mass, and the
	Bishop intones the Jam non dicam from the Pontifical, which the choir
	continues. After the intonation of the Jam non dicam, the gremiale bearer
	places the faldstool in the middle of the predella.

	The Bishop goes to the faldstool with his ministers, where, sitting, he
	receives the miter from the Deacon. At the Gloria Patri all make a profound
	head bow. The book bearer takes the Pontifical from the Missal stand, which
	is replaced by the Missal.

	\section{Credo}

	After the Gloria Patri the new priest positions himself on the floor in
	front of the Bishop, and, after the singing of the Jam non dicam, in a
	clear but subdued voice, recites the Apostles’ Creed, after which the
	Bishop sits.

	\section{The Final Imposition of Hands}

	The book and candle bearers position themselves to the left of the Bishop.

	The new priest, having made the required reverences, ascends to the altar
	and kneels on the edge of the predella. The Bishop, sitting, imposes his
	open hands on the head of the new priest saying Accipe Spiritum Sanctum
	etc.

	\section{Unfolding of the Chasuble}

	Assisted by Mc1 and Mc2, the Bishop loosens the back of the chasuble,
	saying Stola innocentiae etc.

	\section{Pax}

	The Bishop, holding the new priest’s hands in his own, still sitting, moves
	to the right of the new priest, saying Pax Domini sit semper tecum, to
	which the new priest responds Amen.

	The new priest then descends to the foot of the altar, where, having made
	the prescribed reverences, returns to his kneeler and kneels.

	\section{Exhortation}

	The Bishop, having received the crozier, reads the instruction Quia res
	etc.

	\section{Final Blessing of the New Priest}

	The Bishop rises, and, retaining the miter and crozier, says recto tono
	Benedictio Dei etc., blessing the new priest who is kneeling before him.
	The ministers respond Amen and the book and candle bearers depart. The
	Bishop sits, gives away the crozier and has the miter removed.

	\section{Communion Verse}

	The Assistant Priest awaits the Bishop at the Missal at the Epistle corne
	of the altar. The Bishop rises and goes to the Missal at the Epistle
	corner, where he reads the Communion verse together with the new priest.

	When the Bishop goes to the Epistle corner, the gremiale bearer removes the
	faldstool from the predella, lest it impede the rites that follow.

	\section{Postcommunion Prayers}

	The Bishop sings Dominus vobiscum and the Postcommunion prayer of the day
	and then a prayer for the new priest under one conclusion, while the new
	priest reads them simultaneously in a low voice.

	\section{Ite Missa Est}

	The Deacon, and he alone, sings the Ite Missa est in the usual manner.

	\section{Pontifical Blessing}

	The Bishop, having received the miter, solemnly imparts the Pontifical
	Blessing in the usual manner, receiving the crozier before he says Patris.
	The new priest, although he is concelebrating, receives the blessing
	kneeling and signs himself.

	\section{Final Exhortation and Penance}

	After the Pontifical Blessing has been imparted, the Bishop moves to the
	Epistle side of the altar while the gremiale bearer places the faldstool in
	the middle of the altar. The Bishop sits, retaining the miter and crozier.

	The book and candle bearers position themselves to the left of the Bishop,
	who then

	reads the exhortation Fili dilectissime, and then imposes the penance on
	the new priest, who responds Libenter .

	\section{The Last Gospel}

	The Bishop, once the miter has been removed, rises and goes to the Gospel
	corner, where he says together with the new priest Dominus vobiscum, and
	then reads the Last Gospel of St. John, which is also to be recited by the
	new priest, and the faldstool is removed by the gremiale bearer.

	The Bishop, after reading the last Gospel, descends to the floor with his
	ministers, and, having made a profound reverence to the altar, and having
	received the precious miter and crozier, returns to the sacristy,
	accompanied by his ministers.

	\section{Unvesting}

	The sacred ministers lay aside their maniples. After the Bishop has laid
	aside his vestments, the sacred ministers retire, having made the usual
	reverence to the Bishop.

	\section{Thanksgiving and Departure of the Bishop}

	The book and candle bearers kneel before the Bishop while he says the
	thanksgiving prayers.

	}

\end{document}
